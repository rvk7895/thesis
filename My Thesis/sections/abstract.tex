In the era of digital intimacy, online dating applications like Bumble have significantly reshaped social interactions and romantic engagements, particularly in India. This thesis delves into the gender disparities manifested within these platforms, focusing on the algorithmic biases present in Bumble's match recommendations. By examining how these algorithms might favour specific user demographics over others, this research addresses the broader implications of such biases on the visibility and representation of non-binary and marginalized genders. The thesis posits that while these platforms claim neutrality, their underlying algorithms often replicate societal biases, thereby influencing user experiences based on gender. This investigation is anchored in India's rapidly evolving digital landscape, where traditional norms meet a burgeoning digital dating culture.

Employing a mixed-methods approach, this study combines quantitative data analysis with qualitative interviews to offer a comprehensive view of how users perceive and experience gender biases. The quantitative analysis assesses the disparities in match recommendations. At the same time, qualitative insights are gathered from user narratives to explore the subjective impact of these algorithms on individual self-perception and interpersonal interactions. This methodological framework highlights the prevalence of algorithmic bias and enriches the understanding of its practical effects on users. The research findings indicate a significant difference in the algorithmic treatment of users based on gender, revealing a complex interplay of technology and social identity that often reinforces traditional gender roles and stereotypes.

The conclusions drawn from this study underscore the urgent need for algorithmic transparency and the integration of ethical considerations in the design and deployment of dating applications. By documenting the nuanced ways in which gender disparities are embedded within digital platforms, this thesis contributes to the critical discourse on digital ethics and algorithmic fairness. The research advocates for more inclusive and equitable design practices that consider the diverse spectrum of gender identities, ultimately aiming to inform developers, policymakers, and the broader community about the challenges and opportunities for fostering inclusivity in digital spaces. The findings pave the way for future research into the inclusivity of digital platforms and serve as a call to action for adopting more responsible technology practices in online dating.