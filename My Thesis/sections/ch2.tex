This chapter delineates the methodology adopted in this thesis to explore gender disparities in Bumble’s match recommendations. The research design incorporates qualitative, quantitative, and mixed methods to furnish a comprehensive analysis of algorithmic biases \cite{Kalra_Gupta_Varghese_Rangaswamy_2023}. This chapter explicates the procedural framework utilized for participant selection, ensuring a representation that mirrors the diverse demographic of Bumble users, crucial for the validity of the findings \cite{Mislove_Viswanath_Gummadi_Druschel_2010}.

Data were collected through a combination of surveys, structured interviews, and meticulous profile analysis, with each method carefully selected to address specific research questions \cite{Kalra_Gupta_Varghese_Rangaswamy_2023}. This multimodal approach enables a thorough investigation into the algorithmic processes and their implications on user experience and bias reproduction. Ethical considerations are paramount, guiding all data collection activities to ensure compliance with research ethics standards and participant privacy \cite{Elovici_Fire_Herzberg_Shulman_2014}.

The data analysis procedures employed are detailed subsequently, highlighting the analytical techniques and statistical methods used to dissect the complex interplay between user behavior and algorithmic recommendations. This robust analytical approach allows for nuanced insights into how gender biases may manifest in digital matchmaking environments \cite{Raghavan_Barocas_Kleinberg_Levy_2019}.

The chapter concludes with a discussion on the limitations of the research design and potential biases that may influence the findings. Strategies to mitigate these limitations are proposed, emphasizing the need for transparency and reflexivity in algorithmic research. The chapter sets the stage for the subsequent chapters, which delve into the empirical findings and theoretical implications of the study.

\section{Research Design}
\subsection{Selection of Research Methodology}
This study employs a mixed-methods approach, integrating both qualitative and quantitative research methodologies to provide a robust analysis of the algorithmic biases present in Bumble's matchmaking system. The rationale for adopting a mixed-methods approach is to capitalize on the strengths of both qualitative and quantitative research in uncovering nuanced insights into the algorithmic processes and their effects on different genders. Mixed methods allow for a comprehensive understanding of both the measurable trends through quantitative analysis and the subjective experiences through qualitative insights \cite{Kalra_Gupta_Varghese_Rangaswamy_2023}. 

\subsection{Quantitative Research Component}
The quantitative aspect of this research primarily revolves around the use of surveys and profile analysis. This component is crucial for obtaining measurable data on user interactions and algorithmic output, which can be statistically analyzed to detect patterns of bias or discrepancies in match recommendations across different gender identities.

\begin{itemize}
    \item \textbf{Surveys:} Structured surveys were administered to a diverse group of Bumble users to gather data on user satisfaction and perceived fairness in match recommendations. The surveys included both closed and open-ended questions, allowing for the collection of both quantitative data and qualitative feedback.
    \item \textbf{Profile Analysis:} An analytical review of user profiles was conducted to quantitatively assess how the algorithm categorizes and prioritizes user profiles. This involved collecting data on factors such as the frequency of matches, user preferences, and the diversity of profile recommendations. 
\end{itemize}

\subsection{Qualitative Research Component}
The qualitative component focuses on interviews and thematic analysis, providing deeper insights into the personal experiences of users and their perceptions of algorithmic bias.

\begin{itemize}
    \item \textbf{Interviews:} Semi-structured interviews were conducted with selected Bumble users who have experienced varying levels of interaction and success on the platform. These interviews helped to uncover personal narratives that illustrate the impact of algorithmic decisions on real-world social interactions and relationships.
    \item \textbf{Thematic Analysis:} Responses from interviews and open-ended survey questions were subjected to thematic analysis to identify common themes and patterns that may not be immediately evident through quantitative methods alone.
\end{itemize}

\subsection{Integration of Research Components}
The integration of quantitative and qualitative components in this mixed-methods design facilitates a more comprehensive understanding of both the statistical and humanistic aspects of algorithmic bias in dating apps. This approach not only provides a broad statistical overview of user experiences but also delves deeply into individual stories and perspectives, enriching the interpretation of data with personal contexts and nuanced insights.


This mixed-methods research design is meticulously crafted to explore the intricate dynamics of algorithmic matchmaking in a popular online dating platform. By combining quantitative data analysis with qualitative inquiries, the study aims to provide a holistic view of the algorithmic biases that potentially affect gender disparities in match recommendations. The methodological rigor and interdisciplinary approach ensure that the findings contribute valuable insights into the ongoing discussions about fairness and inclusivity in digital environments.

\section{Participant Selection and Demographics}
In this study, a purposive sampling technique was employed to select participants who were active or past users of Bumble, ensuring a focus on individuals directly impacted by the matchmaking algorithms of the platform. This method of sampling is critical for qualitative research where the need for specific types of information arises, and where the researcher seeks to understand common themes across a specific group of people.

\subsection{Criteria for Participant Selection}
Participants were selected based on several criteria:

\begin{itemize}
    \item \textbf{Age:} All participants were within the age group of 18-24 years, aligning with Bumble’s primary user demographic in urban areas like Mumbai, where the study was focused.
    \item \textbf{Usage:} Both current and past users of Bumble were included to gain perspectives from active users and those who may have discontinued use due to dissatisfaction or other reasons.
    \item \textbf{Gender Identity:} The study targeted a balanced representation across different genders, including cisgender men and women, as well as non-binary individuals. This was crucial to explore the differential impacts of algorithmic biases across these groups.
\end{itemize}

\subsection{Demographic Breakdown}
The participant pool for the profile analysis consisted of:
\begin{itemize}
    \item Two self-idenitfied men
    \item Two self-idenitfied women
    \item One self-idenitfied non-binary person
\end{itemize}

The participant pool for the interviews included:
\begin{itemize}
    \item Three self-identified cis-men
    \item Two self-identified bisexual-men
    \item Three self-idenitfied cis-women
    \item One self-idenitfied bisexual-woman
    \item One self-idenitfied bisexual non-binary person
\end{itemize}
This demographic mix was instrumental in exploring the nuances of how gender impacts user experience on Bumble, particularly in the context of algorithmic matchmaking.

Mumbai was selected as the primary location for participant recruitment. Mumbai was chosen due to its high urban population and technological savviness, factors which contribute to a diverse user base with varied dating preferences. This diversity is essential for examining the complex dynamics of Bumble's matchmaking algorithm and its adaptability to different user profiles. Mumbai's cultural richness and demographic diversity provide a unique context for exploring how cultural and societal norms influence dating behaviors and attitudes towards the LGBTQIA+ community. These factors are critical in understanding how users interact with the app and each other, potentially affecting the algorithm's function and effectiveness in making match recommendations. The choice of Mumbai impacts the study by introducing specific cultural and demographic variables that may influence user behavior and algorithmic responses. These include varying levels of social acceptance, stigma towards different gender identities, and the overall openness of the community to online dating. Such factors are anticipated to play a significant role in shaping the data collected, particularly concerning how different genders experience the platform.

\subsection{Ethical Consideration}
In recruiting participants, ethical considerations were strictly adhered to, with all participants providing informed consent. The research design included protocols to protect participant anonymity and confidentiality, which is particularly important given the sensitive nature of data related to personal relationships and dating behaviors. Ethical guidelines from the International Institute of Information Technology, Hyderabad, were followed to ensure the study’s integrity and the welfare of all participants \cite{Kalra_Gupta_Varghese_Rangaswamy_2023}.

\subsection{Additional Considerations}
To deepen the understanding of the dataset and enhance the robustness of the study, it would also be beneficial to consider:

\begin{itemize}
    \item \textbf{Socioeconommic Status:} Future research could include participants from varied socioeconomic backgrounds to examine if and how socioeconomic factors influence user experiences and interactions with the Bumble algorithm.
    \item \textbf{Demographic Status:} While this study focused on Mumbai, expanding the research to include other cities or rural areas could provide insights into regional variations in algorithmic impact, reflecting broader demographic trends and preferences.
\end{itemize}


The selection of participants and the demographic characteristics are pivotal in research as they directly influence the reliability and applicability of the findings. A well-rounded demographic profile supports the generalization of the study results to a broader population and ensures that the conclusions drawn are reflective of diverse user experiences, thereby enhancing the study’s credibility and relevance in discussions on algorithmic fairness in social platforms.

This approach not only fulfills the research’s methodological needs but also aligns with broader ethical and scholarly standards, ensuring that the study’s outcomes can effectively contribute to ongoing debates and future policymaking in the realm of digital dating and algorithmic mediation.

\section{Data Collection Methods}
This study utilized a comprehensive multi-method approach to collect data, employing surveys, interviews, and profile analysis to capture both quantitative and qualitative insights into user experiences and algorithmic performance on Bumble \cite{Kalra_Gupta_Varghese_Rangaswamy_2023}. Each method was chosen to complement and enhance the insights gained from the others, ensuring a robust analysis of the algorithms’ impact across different demographic groups.

\subsection{Surveys}
Structured surveys were administered to a diverse group of Bumble users to quantitatively assess their satisfaction and perceived fairness in match recommendations. These surveys included a mix of scaled responses and open-ended questions to facilitate quantitative analysis of user satisfaction and qualitative insights into user perceptions and experiences.

\begin{itemize}
    \item \textbf{Design and Distribution:} The surveys were designed to be comprehensive yet concise, ensuring a high completion rate. They were distributed electronically via email and social media platforms, targeting users from the demographic groups identified in the participant selection phase. 
    \item \textbf{Content:}  Questions focused on user satisfaction with the matches received, perceived biases in the recommendation system, and personal impact of these biases. This included asking participants to rate their satisfaction on a Likert scale and provide narrative descriptions of their experiences.
\end{itemize}

\subsection{Profile Analysis}
Profile analysis in this study was meticulously designed to quantitatively examine the operation of Bumble’s matchmaking algorithm, focusing particularly on the algorithm's handling of diverse gender identities. This analysis aimed to uncover potential biases and operational patterns by observing the profiles shown to users and the interactions these profiles elicited. The decision to focus particularly on non-binary genders stemmed from a gap identified in existing research, which often centers around the binary genders of male and female. Non-binary individuals frequently face unique challenges and biases on dating platforms, which can be overshadowed in broader algorithmic studies. By concentrating on this demographic, the study sought to illuminate the specific ways in which non-binary profiles are treated differently by the algorithm, providing insights into inclusivity and fairness of the platform. Bumble allows users to select from an extensive list of gender identities, which includes but is not limited to Man, Woman, Non-Binary, Transgender Man, Transgender Woman, Genderfluid, Genderqueer, Gender Non-conforming, and Two-Spirit. We made a conscious choice of non-binary to be less specific about gender preference and also to adhere to the conventions of  Bumble, which showcases the major genders as man, woman and non-Binary. 

\begin{itemize}
    \item \textbf{Data Collection:} Data was captured on various attributes of the profiles recommended to users, including age, gender, location, and interests.
    \item \textbf{Analysis:} The analysis focused on understanding the correlations between user preferences (as indicated in their profiles and swipe behavior) and the characteristics of the profiles recommended to them. Special attention was given to discrepancies in recommendations across different genders.
\end{itemize}

The data displayed on each user profile is selected voluntarily by the individual when setting up their account on Bumble. Users have the discretion to include or omit various details, such as age, occupation, and location, among others. While not mandatory, filling out these profile attributes is encouraged by Bumble, as the platform suggests that a more complete profile enhances the algorithm's ability to match users with potential suitors who are likely to be more compatible. This feature underlines the balance between user privacy and the effectiveness of the matchmaking process, suggesting that richer profile information can lead to more meaningful connections.

To supplement the observational data from surveys and profiles, an experimental design could also be employed where user interactions are simulated under controlled conditions to observe the outcomes of specific changes in the algorithm. This method would allow researchers to isolate specific factors and directly measure the algorithm’s responsiveness to controlled variations in user behavior and profile settings.

With the latest updates to Bumble's mobile application, the methodology employed in the profile analysis requires an update to account for these changes. The previous analysis conducted on the web version of Bumble required non-binary users to select whether they wanted their profiles to appear in searches for men or women. However, the mobile application has now eliminated this requirement, allowing non-binary users to be displayed without having to categorize themselves within the binary gender options. This significant platform update necessitates a reevaluation of the research methodology and a comparative analysis to determine if these application changes affect the user experience and algorithmic output for non-binary users.

\subsection{Interviews}
Interviews in this study were designed to gather in-depth qualitative data from Bumble users, providing a deeper understanding of individual experiences and perceptions regarding the app's matchmaking algorithm. The semi-structured nature of these interviews allowed for flexibility, letting participants freely express their thoughts while still guiding the conversation to cover specific topics of interest essential to the research questions. Participants for the interviews were selected based on their responses to the initial survey. Those who indicated particularly insightful or unique experiences with Bumble—such as perceived biases or significant impacts on their dating life—were chosen to provide a richer, more detailed account of their interactions with the platform.

The interview process was as follows:
\begin{itemize}
    \item \textbf{Preparation:} Ahead of the interviews, a guide was prepared with open-ended questions to explore themes like user satisfaction, perceived fairness, and personal stories of match interactions. Example questions included:
    \begin{itemize}
        \item "Can you describe any patterns you've noticed in the matches you receive?"
        \item "How do these patterns align with your expectations and preferences?"
        \item "Have you ever felt that your profile was treated differently compared to others? Please elaborate."
    \end{itemize}
    \item \textbf{Conducting Interviews:} Interviews were conducted via video calls to accommodate the participants' geographical diversity and recorded with their permission for accuracy in data collection. Each session lasted approximately 30-45 minutes.
    \item \textbf{Data Handling:} All interviews were transcribed verbatim, and identifying information was anonymized to maintain confidentiality. The transcripts were then analyzed using thematic analysis to identify common themes and significant outliers in user experiences.
\end{itemize}

\section{Ethical Concerns}
In this section, we will first attempt to identify the potential ethical concerns that can arise while conducting this experiment. The ethical issues that are relevant to us are the ones that are morally wrong things or can any problematic outcomes that can be a result of this experiment. After identifying, we will either attempt to justify how an ethical concern can be ignored or explain what steps we are taking to counter these ethical concerns. The potential ethical issues that we see are as follows:
\begin{itemize}
	\item Informed consent: It is essential to ensure that all participants fully understand the nature and purpose of the study and any potential risks or benefits before they agree to participate. This means that the researchers must provide clear and comprehensive information about the study and ensure that participants can ask questions and raise concerns.
	\item Confidentiality: The researchers must ensure that the personal information of the participants is kept confidential and secure. This includes any data collected about the profiles recommended to them and any information about their profiles and interactions on the Bumble platform.
	\item Deception: The use of fake profiles in this study may be seen as a form of deception, as the participants' actual gender will not be disclosed on one of their profiles. This could be unethical, as it may cause some participants to feel misled or manipulated. To mitigate deception we will contact/messate a Bumbel user who is matched with our participant profile mentionign our research project and goals- we will take necessary permissions to use their profile data, [ may of which which is anyways public on Bumble] If the user is reluctant we will refrain from usign any data from this particular user.
	\item Risk of harm: There is a potential risk of harm to the participants if they are subjected to harmful or abusive interactions due to their participation in the study. The researchers should minimize this risk by providing participants with support and resources if they experience harmful or abusive interactions.
	\item Privacy: The participants' privacy must be respected and protected throughout the study. This means that the researchers must ensure that any data collected about the participants is used only for the study and is not shared with any third parties without the participant's consent.
\end{itemize}

Among these concerns, Informed Consent, Deception, and Risk of Harm come up because of the use of fake profiles. However, as the users have given their explicit consent to show the data they want to display to other users on Bumble, we can collect the data that users have themselves consented to share. We can look at other studies that have collected data using fake profiles. Mislove et al. in \cite{Mislove_Viswanath_Gummadi_Druschel_2010} use fake profiles on Facebook to collect data about university students. Facebook does not provide an open API giving access to data about its users. Thus the data collected was done using web scraping methods by logging into a Facebook profile and performing a Breadth-First Search on the accessible users from that profile.

Jernigan and Mistree, in their study \cite{Jernigan_Mistree_2009} conducted in 2009, used a web scraping bot named Arachne to collect information about Facebook users. The bot would log in to Facebook, receive cookies from Facebook, and download Web pages with profile and friend information. They conducted this without the permission of Facebook.

Yuval Elovici et al. in \cite{Elovici_Fire_Herzberg_Shulman_2014} introduce the concept of Public Information and Private Information. They mention that Public information can be acquired without establishing any form of connection between the two users, i.e., on Facebook, without becoming friends with the other user, and only collecting the data displayed without becoming friends with them. In the case of Bumble, this is the data shown on the user's profile that is displayed to others users, based on which the user decides whether they right swipe on the person or left swipe. This is the data we would be collecting from the public information domain.

Elovici et al. in \cite{Elovici_Fire_Herzberg_Shulman_2014} also defined Private Information in Online Social Networks. Private information in OSN is defined as the information only available to the users' friends \cite{Elovici_Fire_Herzberg_Shulman_2014}. In the case of Facebook, Private Information is trivial to understand. In the case of Bumble, we can understand Private information as the information that a user can acquire when they are matched with a potential suitor. This information includes all the information that is possible to gain via communicating with the other user using the methods provided by Bumble, which include Text Chat, Voice Call, and Video Call. As our experiment does not involve the users interacting with each other, we will not collect any form of private information about any of the users. Thus the risk of harm is minimized.

If the profile were to get a match with another user, we would inform the user that this profile was made purely for research purposes, and there would be no interaction from their side. We will also ask for their explicit consent on whether they want their profile to be part of the study. 

Continuing on the thread of risk of harm, concerns can also be raised about the psychological harm for the potential suitors, the users who are legitimately using the platform to find relationships. To minimize any form of psychological harm to the users because of the fake profile we would be employing the following methods:
\begin{itemize}
    \item Informed Consent: It is critical to guarantee that all actual users are informed of the study and that phoney profiles may be utilised as part of the research. This can assist to guarantee that users are not taken aback when they discover they have been communicating with a bogus profile.
	\item Minimize the risk of deceptions: To reduce the possibility of deceit, we may include clear and conspicuous disclaimers on the fake profiles to indicate that they are part of a research and ensure that the profiles do not contain any inaccurate or misleading material.
\end{itemize}
Concerns regarding the privacy of the users whose data is being collected are also to be considered. Our study does not involve sharing data with any public platforms. All the data that would be collected would first be anonymized and then stored securely offline to prevent any form of data leaks. If we allow other users to access information from the data, it would be released securely by applying Differential Privacy algorithms to release statistical information about the dataset.

The study’s protocol was reviewed and approved by the ethics committee at the International Institute of Information Technology, Hyderabad. This review ensured that all planned research activities were conducted ethically and in compliance with both institutional and international standards.
%Should we cite this?

\section{Limitations and Delimitations}
\subsection{Delimitations}
Delimitations refer to the choices made during the research design that define the scope and boundaries of the study. These are decisions made by the researchers to narrow the focus of the study intentionally.

\begin{itemize}
    \item \textbf{Geographic Scope:} The study was confined to users based in Mumbai, India. While this allowed for a detailed examination of the algorithm’s impact within a specific urban context, it may not be generalizable to rural areas or other urban contexts both within and outside of India.
    \item \textbf{Age and Gender:} The study focused on a relatively narrow age range (18-24 years) and included primarily cisgender and non-binary individuals. The decision to focus on these demographics was driven by the need to explore specific issues within a manageable and relevant sample but limits the applicability of findings to older age groups or other gender identities not included in the study.
    \item \textbf{Data Collection Methods:} The reliance primarily on surveys, interviews, and profile analysis might not capture all dimensions of user experience or algorithm functionality. Alternative methods, such as direct observation or log file analysis, were outside the scope of this study.
\end{itemize}

\subsection{Limitations}
Limitations are factors that the researchers did not have control over, which might influence the interpretation and application of the findings.

\begin{itemize}
    \item \textbf{Sample Size and Selection:} While purposive sampling was appropriate for the exploratory nature of this study, the sample size and method limit the generalizability of the findings. The participants might not represent all Bumble users, especially considering the diverse global user base.
    \item \textbf{Self-Reporting Bias:} The data collected through surveys and interviews are subject to self-reporting biases. Participants may have provided responses they deemed socially acceptable or that reflect self-perception rather than objective reality.
    \item \textbf{Algorithm Transparency:} The proprietary nature of Bumble’s matchmaking algorithm means that the research could not directly analyze the algorithm’s code or operational mechanics. This limits the ability to definitively pinpoint how the algorithm processes different gender identities or user behaviors. 
\end{itemize}

\subsection{Mitigation Strategies}
To mitigate some of these limitations, several strategies were employed:
\begin{itemize}
    \item \textbf{Triangulation of Data Sources:} By using multiple methods of data collection, the study aimed to corroborate findings across different data points, reducing the impact of biases associated with any single method.
    \item \textbf{Statistical Controls:} Where possible, statistical techniques were used to control for confounding variables, enhancing the reliability of the findings.
    \item \textbf{Transparent Reporting:} The research provides a transparent account of the methodologies, analytical techniques, and challenges encountered, which helps in the critical evaluation of the study’s findings and methodology.
\end{itemize}

